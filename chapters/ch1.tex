\chapter{Basic mechanics}

\newcommand{\shp}[3]{\item \textbf{#1} as in #2 (#3)}

\section{Phonology}

The word in parentheses at the end denote the Necarasso Cryssesa (thereafter abbreviated as NCS) words for the respective letters (which are also regular words).

\begin{itemize}
  \shp{c}{cat}{cvyssalyr}
  \shp{e}{egg}{eltes}
  \shp{n}{neck}{nesmeria}
  \item \textbf{v} is similar to the English counterpart (as in vine), except that it is pronounced with the two lips touching each other, instead of the upper teeth touching the lower lip. (vystos)
  \item \textbf{o} has no example in English. It is similar to the \emph{o} in \emph{own}, but with no glide (i.~e.~with only the first part, no ``ooh''). (oscona)
  \shp{s}{sit}{senar} (note: can be pronounced \emph{z} as well)
  \shp{r}{rat}{roton}
  \shp{l}{leg}{lyre}
  \shp{m}{melt}{mioros}
  \shp{a}{box}{arcyn}
  \item \textbf{f} is similar to the English counterpart (as in fox), except that it is pronounced with the two lips touching each other, instead of the upper teeth touching the lower lip. (fyrno)
  \shp{g}{gate}{gasyda}
  \shp{p}{pit}{perselta}
  \shp{t}{tab}{tesenor}
  \shp{i}{seed}{iplymos}
  \item \textbf{y} is similar to \textbf{i}, but is shorter. (ynoros)
  \shp{d}{den}{decso}
  \item \textbf{h} sounds similar to the English counterpart, but the sound is produced in the back of the mouth (same area as \textbf{c} or \textbf{g}) instead of in the throat.
  \item \textbf{ss} as in thin
  \item \textbf{ll} is pronounced like \textbf{s}, but the air flows to the sides of the tongue.
  \item \textbf{css} consists of pronouncing \textbf{h} and \textbf{ss} simultaneously.
\end{itemize}

\textbf{Notes:}

\begin{itemize}
  \item All unvoiced consonants are aspirated; in other words, \textbf{t} is pronounced as the one in \emph{top} instead of \emph{stop}, even when there is an \textbf{s} next to it.
  \item A \textbf{d} at the end is pronounced as \textbf{t}.
  \item \textbf{i} is always long (except when the next bullet point applies) and \textbf{y} is always short. Other vowels are long if and only if they precede another vowel, \textbf{r}, or \textbf{ll}, or if they occur at the end of a word. \textbf{e} and \textbf{o} is pronounced with a more open mouth when long.
  \item If \textbf{i} follows another vowel (except \textbf{y}), then it is pronounced as the \emph{y} in \emph{yet}. Note that the sequence \textbf{ri} before another vowel is pronounced as \emph{y} as well. If the vowel following it is umlauted, then the \textbf{i} is pronounced normally.
  \item Stress is:
  \begin{itemize}
    \item On the last syllable if the word ends in a \textbf{d} or \textbf{l}
    \item On the last umlauted vowel if any exists
    \item On the antepenultimate syllable if it is long and the penultimate is short
    \item On the penultimate syllable otherwise
  \end{itemize}
  \item If \textbf{r} occurs both before and after a vowel, the second one is not pronounced.
  \item The only permitted endings (accounting for palatalization) are -a, -e, -i, -o, -as, -es, -is, -os, -ys, -an, -en, -on, -yn, -ia, -io, -ias, -ios, -ian, -ion, -ass, -ess, -yss, -erss, -el, and -yl, with -os, -ios, -on, -ion, -or, and -el masculine.
\end{itemize}

\textbf{Pronounce each of the words carefully.}

\begin{enumerate}
  \item \textbf{vercesa} (grain, fleck)
  \item \textbf{ergelyd} (to admire)
  \item \textbf{avona} (wind, air, gas)
  \item \textbf{retyrcar} (flower)
  \item \textbf{myron} (after)
  \item \textbf{mortos} (hand)
  \item \textbf{arpelia} (stream)
  \item \textbf{nari\"a} (chin)
  \item \textbf{ismel} (borax)
  \item \textbf{csserys} (door, gate)
  \item \textbf{rialad} (to lead)
  \item \textbf{enlea} (far)
\end{enumerate}

\section{Nouns}

Nouns are conceptually identical to the English counterparts. However, they are inflected for three numbers instead of two in English. \\

In English, nouns can be singular or plural -- the former meaning one of a thing and the latter more than one. NCS also has the dual number, which signifies two of an object.

\begin{center}
  \begin{tabular}{|l|l|}
    \hline
    \textbf{English} & \textbf{NCS} \\ \hline
    one cat & alarys vyl \\ \hline
    two cats & alarer enefa \\ \hline
    three cats & alarillyr epremo \\ \hline
  \end{tabular}
\end{center}

Notice that NCS duals and plurals are more complex to form, but more regular as well. They depend on the ending.

\begin{center}
  \begin{tabular}{|p{4cm}|p{4cm}|p{4cm}|}
    \hline
    \textbf{Ending} & \textbf{Dual} & \textbf{Plural} \\ \hline
    All with a & -ar & -o \\ \hline
    -el & -or & -ion \\
    -e & -ir & -i \\
    All others with e & -yr & e to y \\ \hline
    -o & -yn & -an \\
    -or & -osor & -el \\
    All others with o & -or & -el \\ \hline
    All with i/y & -er & -illyr \\ \hline
    \textbf{Drop palatalization?} & \textbf{Yes} & \textbf{No, unless ending rules require dropping} \\ \hline
  \end{tabular}
\end{center}

\textbf{Examples.}

\begin{center}
  \begin{tabular}{|l|l|l|l|}
    \hline
    \textbf{Singular} & \textbf{Dual} & \textbf{Plural} & \textbf{Definition} \\ \hline
    vercesa & vercesar & verceso & grain, fleck \\ \hline
    retyrcar & retyrcar & retyrco & flower \\ \hline
    mortos & mortor & mortel & hand \\ \hline
    arpelia & arpelar & arpelio & stream \\ \hline
    cerel & ceror & cerion & sunset \\ \hline
    csserys & csserer & csserillyr & door \\ \hline
    nerdo & nerdyn & nerdan & base, foundation, floor \\ \hline
    creten & cretyr & cretyn & wave \\ \hline
    nari\"a & nari\"ar & nari\"o & chin \\ \hline
  \end{tabular}
\end{center}

\section{Personal pronouns}

Before we continue to the next section, it might be convenient to look at personal pronouns (e.~g.~I, you):

\begin{center}
  \begin{tabular}{|r|l|l|l|}
    \hline
    & \textbf{SG} & \textbf{DU} & \textbf{PL} \\ \hline
    \textbf{1} & e \emph{I} & ento & eras \emph{we} \\ \hline
    \textbf{2} & eo \emph{you} & eoro & eos \\ \hline
    \textbf{3} & os \emph{he} & oson & oros \\
    & er \emph{she} & eren & erys \\
    & an \emph{one} & & \\ \hline
  \end{tabular}
\end{center}

\section{Verbs}

As with nouns, verbs are conceptually identical to their English counterparts. However, their uses are more complex.

First, verbs in NCS are inflected for three persons (first, second, and third) and number (singular, dual, and plural). In addition, they inflect for four \emph{moods}:

\begin{itemize}
  \item \textbf{Indicative} denotes a certain statement (e.~g.~\emph{It snowed yesterday. I gave him the book.}).
  \item \textbf{Subjunctive} denotes an uncertain statement (e.~g.~\emph{I'm not sure whether it will snow tomorrow. I'll give him the book if he \emph{comes to school}.}).
  \item \textbf{Imperative} denotes a command, request, need, or desire (e.~g.~\emph{Please give me the book. You want her to help you. It's important to eat every day.}).
  \item \textbf{Interrogative} denotes a question (e.~g.~\emph{Which book did you receive?}). Unless provided separately, it is inflected identically as the indicative.
\end{itemize}

Verbs are inflected in five patterns (\emph{asagi}; sg.~\emph{asage}): \\

\begin{center}
  \textbf{0 asage.} Ends in \textbf{-ad} but not \textbf{-ead}. \\
  \textbf{cynrad} - to open \\
  \begin{tabular}{|r|l|l|l|}
    \hline
    \textbf{Indicative} & \textbf{SG} & \textbf{DU} & \textbf{PL} \\ \hline
    \textbf{1} & e \textbf{cynra} & ento \textbf{cynran} & eras \textbf{cynress} \\ \hline
    \textbf{2} & eo \textbf{cynres} & eoro \textbf{cynresen} & eos \textbf{cynrer} \\ \hline
    \textbf{3} & os \textbf{cynre} & oson \textbf{cynren} & oros \textbf{cynri} \\ \hline
    \textbf{Subjunctive} & \textbf{SG} & \textbf{DU} & \textbf{PL} \\ \hline
    \textbf{1} & e \textbf{cynrena} & ento \textbf{cynrenera} & eras \textbf{cynreness} \\ \hline
    \textbf{2} & eo \textbf{cynrenes} & eoro \textbf{cynreneras} & eos \textbf{cynrener} \\ \hline
    \textbf{3} & os \textbf{cynrene} & oson \textbf{cynrenera} & oros \textbf{cynreni} \\ \hline
    \textbf{Imperative} & \textbf{SG} & \textbf{DU} & \textbf{PL} \\ \hline
    \textbf{1} & e \textbf{cynrenta} & ento \textbf{cynrenela} & eras \textbf{cynrentess} \\ \hline
    \textbf{2} & eo \textbf{cynrentes} & eoro \textbf{cynrenelas} & eos \textbf{cynrenter} \\ \hline
    \textbf{3} & os \textbf{cynrente} & oson \textbf{cynrenela} & oros \textbf{cynrenti} \\ \hline
  \end{tabular} \\
  \textbf{1 asage.} Ends in \textbf{-yd} but not \textbf{-ayd}. \\
  \textbf{yndaryd} - to leave \\
  \begin{tabular}{|r|l|l|l|}
    \hline
    \textbf{Indicative} & \textbf{SG} & \textbf{DU} & \textbf{PL} \\ \hline
    \textbf{1} & e \textbf{yndare} & ento \textbf{yndaren} & eras \textbf{yndarass} \\ \hline
    \textbf{2} & eo \textbf{yndaras} & eoro \textbf{yndaresan} & eos \textbf{yndarar} \\ \hline
    \textbf{3} & os \textbf{yndara} & oson \textbf{yndaran} & oros \textbf{yndaro} \\ \hline
    \textbf{Subjunctive} & \textbf{SG} & \textbf{DU} & \textbf{PL} \\ \hline
    \textbf{1} & e \textbf{yndarese} & ento \textbf{yndaresere} & eras \textbf{yndaresass} \\ \hline
    \textbf{2} & eo \textbf{yndaresas} & eoro \textbf{yndareseras} & eos \textbf{yndaresar} \\ \hline
    \textbf{3} & os \textbf{yndaresa} & oson \textbf{yndaresera} & oros \textbf{yndareso} \\ \hline
    \textbf{Imperative} & \textbf{SG} & \textbf{DU} & \textbf{PL} \\ \hline
    \textbf{1} & e \textbf{yndarepe} & ento \textbf{yndarepele} & eras \textbf{yndaretass} \\ \hline
    \textbf{2} & eo \textbf{yndaretas} & eoro \textbf{yndareselas} & eos \textbf{yndaretar} \\ \hline
    \textbf{3} & os \textbf{yndareta} & oson \textbf{yndaresela} & oros \textbf{yndareto} \\ \hline
  \end{tabular} \\
  \pagebreak
  \textbf{2 asage.} Ends in \textbf{-ead}. \\
  \textbf{sendread} - to be in excess \\
  \begin{tabular}{|r|l|l|l|}
    \hline
    \textbf{Indicative} & \textbf{SG} & \textbf{DU} & \textbf{PL} \\ \hline
    \textbf{1} & e \textbf{sendrea} & ento \textbf{sendrean} & eras \textbf{sendrehess} \\ \hline
    \textbf{2} & eo \textbf{sendrehes} & eoro \textbf{sendrehesen} & eos \textbf{sendreher} \\ \hline
    \textbf{3} & os \textbf{sendrehe} & oson \textbf{sendrehen} & oros \textbf{sendrei} \\ \hline
    \textbf{Subjunctive} & \textbf{SG} & \textbf{DU} & \textbf{PL} \\ \hline
    \textbf{1} & e \textbf{sendrehena} & ento \textbf{sendrehenera} & eras \textbf{sendreheness} \\ \hline
    \textbf{2} & eo \textbf{sendrehenes} & eoro \textbf{sendreheneras} & eos \textbf{sendrehener} \\ \hline
    \textbf{3} & os \textbf{sendrehene} & oson \textbf{sendrehenera} & oros \textbf{sendreheni} \\ \hline
    \textbf{Imperative} & \textbf{SG} & \textbf{DU} & \textbf{PL} \\ \hline
    \textbf{1} & e \textbf{sendrehenta} & ento \textbf{sendrehenela} & eras \textbf{sendrehentess} \\ \hline
    \textbf{2} & eo \textbf{sendrehentes} & eoro \textbf{sendrehenelas} & eos \textbf{sendrehenter} \\ \hline
    \textbf{3} & os \textbf{sendrehente} & oson \textbf{sendrehenela} & oros \textbf{sendrehenti} \\ \hline
    \textbf{Interrogative} & \textbf{SG} & \textbf{DU} & \textbf{PL} \\ \hline
    \textbf{1} & e \textbf{sendria} & ento \textbf{sendrian} & eras \textbf{sendrehess} \\ \hline
    \textbf{2} & eo \textbf{sendrehes} & eoro \textbf{sendrehesen} & eos \textbf{sendreher} \\ \hline
    \textbf{3} & os \textbf{sendrehe} & oson \textbf{sendrehen} & oros \textbf{sendri} \\ \hline
  \end{tabular} \\
  \textbf{3 asage.} Ends in \textbf{-ayd}. \\
  \textbf{ylmayd} - to panic \\
  \begin{tabular}{|r|l|l|l|}
    \hline
    \textbf{Indicative} & \textbf{SG} & \textbf{DU} & \textbf{PL} \\ \hline
    \textbf{1} & e \textbf{ylmae} & ento \textbf{ylmaen} & eras \textbf{ylmahass} \\ \hline
    \textbf{2} & eo \textbf{ylmahas} & eoro \textbf{ylmaesan} & eos \textbf{ylmahar} \\ \hline
    \textbf{3} & os \textbf{ylmaha} & oson \textbf{ylmahan} & oros \textbf{ylmao} \\ \hline
    \textbf{Subjunctive} & \textbf{SG} & \textbf{DU} & \textbf{PL} \\ \hline
    \textbf{1} & e \textbf{ylmaese} & ento \textbf{ylmaesen} & eras \textbf{ylmaesass} \\ \hline
    \textbf{2} & eo \textbf{ylmaesas} & eoro \textbf{ylmaesenas} & eos \textbf{ylmaesar} \\ \hline
    \textbf{3} & os \textbf{ylmaesa} & oson \textbf{ylmaesan} & oros \textbf{ylmaeso} \\ \hline
    \textbf{Imperative} & \textbf{SG} & \textbf{DU} & \textbf{PL} \\ \hline
    \textbf{1} & e \textbf{ylmaepe} & ento \textbf{ylmaepen} & eras \textbf{ylmaetass} \\ \hline
    \textbf{2} & eo \textbf{ylmaetas} & eoro \textbf{ylmaepenas} & eos \textbf{ylmaetar} \\ \hline
    \textbf{3} & os \textbf{ylmaeta} & oson \textbf{ylmaetan} & oros \textbf{ylmaeto} \\ \hline
    \textbf{Interrogative} & \textbf{SG} & \textbf{DU} & \textbf{PL} \\ \hline
    \textbf{1} & e \textbf{ylmie} & ento \textbf{ylmien} & eras \textbf{ylmahass} \\ \hline
    \textbf{2} & eo \textbf{ylmahas} & eoro \textbf{ylmiesan} & eos \textbf{ylmahar} \\ \hline
    \textbf{3} & os \textbf{ylmaha} & oson \textbf{ylmahan} & oros \textbf{ylmio} \\ \hline
  \end{tabular} \\
  \pagebreak
  \textbf{4 asage.} \emph{Essyd} (to exist) and \emph{ollyd} (+adj, to be) only. \\
  \textbf{essyd} - to exist \\
  \begin{tabular}{|r|l|l|l|}
    \hline
    \textbf{Indicative} & \textbf{SG} & \textbf{DU} & \textbf{PL} \\ \hline
    \textbf{1} & e \textbf{ve} & ento \textbf{ven} & eras \textbf{veass} \\ \hline
    \textbf{2} & eo \textbf{ves} & eoro \textbf{vesen} & eos \textbf{vellar} \\ \hline
    \textbf{3} & os \textbf{vella} & oson \textbf{vellan} & oros \textbf{von} \\ \hline
    \textbf{Subjunctive} & \textbf{SG} & \textbf{DU} & \textbf{PL} \\ \hline
    \textbf{1} & e \textbf{vese} & ento \textbf{vesen} & eras \textbf{vehesass} \\ \hline
    \textbf{2} & eo \textbf{vesas} & eoro \textbf{vesenes} & eos \textbf{vellesar} \\ \hline
    \textbf{3} & os \textbf{vellesa} & oson \textbf{vellesan} & oros \textbf{veson} \\ \hline
    \textbf{Imperative} & \textbf{SG} & \textbf{DU} & \textbf{PL} \\ \hline
    \textbf{1} & e \textbf{vepe} & ento \textbf{vepen} & eras \textbf{vehetass} \\ \hline
    \textbf{2} & eo \textbf{vetas} & eoro \textbf{vepenes} & eos \textbf{velletar} \\ \hline
    \textbf{3} & os \textbf{velleta} & oson \textbf{velletan} & oros \textbf{veton} \\ \hline
    \textbf{Interrogative} & \textbf{SG} & \textbf{DU} & \textbf{PL} \\ \hline
    \textbf{1} & e \textbf{ce} & ento \textbf{cen} & eras \textbf{ceass} \\ \hline
    \textbf{2} & eo \textbf{ces} & eoro \textbf{cesen} & eos \textbf{cellar} \\ \hline
    \textbf{3} & os \textbf{cella} & oson \textbf{cellan} & oros \textbf{gon} \\ \hline
  \end{tabular} \\
  The inflections for \emph{ollyd} are similar, but the indicative forms start with \emph{s}, the subjunctive and imperative forms start with \emph{t}, and the interrogative forms are identical to the indicative.
\end{center}